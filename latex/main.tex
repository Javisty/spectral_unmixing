\documentclass{article}
% The preceding line is only needed to identify funding in the first footnote. If that is unneeded, please comment it out.

\usepackage[preprint]{jmlr2e}

\usepackage{amsmath,amssymb,amsfonts}
\usepackage{algorithmic}
\usepackage{graphicx}
\usepackage{textcomp}
\usepackage{xcolor}

\DeclareMathOperator*{\argmax}{arg\,max}
\DeclareMathOperator*{\argmin}{arg\,min}

\binoppenalty=\maxdimen
\relpenalty=\maxdimen

\def\BibTeX{{\rm B\kern-.05em{\sc i\kern-.025em b}\kern-.08em
    T\kern-.1667em\lower.7ex\hbox{E}\kern-.125emX}}

% Short headings should be running head and authors last names

\ShortHeadings{Learning with Mixtures of Trees}{Côme \& Thouvenin}
\firstpageno{1}

\begin{document}

\title{Learning with Mixtures of Trees}

\author{\name Aymeric CÔME \email aymeric.come.etu@univ-lille.fr \\
       \addr Master Data Science\\
       University of Lille, Centrale Lille, IMT Lille-Douai\\
       Villeneuve d'Ascq, France
       \AND
       \name Pierre-Antoine THOUVENIN \email pierre-antoine.thouvenin@centralelille.fr \\
       \addr Assistant professor and member of SigMA team\\
       Université de Lille, CNRS, Centrale Lille, CRIStAL\\
       Villeneuve d'Ascq, France}

\maketitle

\begin{abstract}%
  Hyperspectral unmixing is the task of infering the abundance of pure materials in an image from the observed spectra. The resulting spectrum indeed depends on some characteristical features of the End-Members (EMs), however this is a challenging problem because not only it is nonlinear, but also miscellaneous factors can change the output, which is known as Spectral Variability (SV). To tackle this problem, a paper \citep{janiczek_differentiable_2020} introduced a physics-based, differentiable model to realistically capture EM HV, and then used it in an optimization framework in order to find promising results for spectral unmixing. However, while the dispersion model seems fruitful, the optimization part raises questions. In the following is presented how we try to improve the unmixing process.
\end{abstract}

\begin{keywords}
Hyperspectral Unmixing, Spectral Unmixing, Alternating Minimization, Proximal
\end{keywords}

\section{Introduction}
\textbf{Context:} what is spectral unmixing, some applications (remote sensing)\\
\textbf{Variability models:} quick description of explicit (additive/multiplicative), black-box (Deep) and physics models.\\
\textbf{Contribution:} take advantage of the dispersion model, with more prior information (regularization) than in Janiczek paper to obtain a more relevant model, while keeping a differentiable problem for an easier resolve (cf natural model is ugly).\\

\textbf{First draft:} In many real-world applications like space exploration, medicine, environment monitoring, mineral detection, hyperspectral images (HI) (consisting of samples of electromagnetic spectra at various wavelengths) can put into evidence some key characteristics of the location monitored. In particular, it is commonly supposed that each present EM have its own signature spectrum, which contributes to the HI depending on its abundance in each pixel. Hence, being able to unmix these characteristic signatures from HI leads to a precise detection of materials in a scene.

However, despite the high spectral resolution of hyperspectral cameras, there are technical limits that must be taken into account; in particular low spatial resolution is to be expected, which leads to a mixture of EMs in each piel. While many unmixing algorithms already have been proposed, the physics behind is often too complex to be fully modelled.

A first difficulty to handle is the SV: depending on the scene observed, an EM might have a different signature than what we were expecting from previous observations. This variability can even occurs locally, between pixels of a same HI, which actually question the hypothesis of one signature corresponding to one EM, rather than a class of signatures. Multiple factors can contribute to this variability, and we distinguish two kinds: intrisic and external. Intrisic variability is due to the natural differences of the EM samples across the HI (e.g. grass is greener on the other side because the neighbor water it) while external variability refers to the environment influence, like light and shade or the quality of the hyperspectral camera used.

In the end, the unmixing problem is a very complex one, and requires advanced technics to overcome the difficulties. We will present some of them in the following section.

\section{Variability models}
Review of the literature.

\subsection{Explicit models}
Learn variability from an a priori structure. Borsoi 2020

\subsection{Deep models}
Learn variability model directly from data, using deep structures. Borsoi 2019 (generative), Hong 2021, Ozkan 2019 (convolution)

\subsection{Physics-based models}\label{sec:dispersion}
Rely on physics properties to infer (complex) models of variability. Try to keep it differentiable. Janiczek 2020.


\section{Problem statement}
Based on the elements and observations we have seen so far, we come up with a formulation for the problems we are willing to tackle.

\subsection{Reconstruction model}
The first part of a spectral unmixing strategy is to try to reconstruct the target HI from a specified model.

Our approach consists in making use of the dispersion model presented in Section \ref{sec:dispersion}, within what we hope to be a better framework. So, the reconstructed spectra are generated by the dispersion model from the physics parameters. More precisely, we suppose that each EM is present in a state characterized by those parameter values in every pixel; and then each EM is associated to a signature spectrum derived from the dispersion model. The SV is modelized here by the variation of the parameter values through the HI, which in turn gives different signature spectra over the pixels. Regarding the unmixing, we decided to use Linear Unmixing once again, so that our model would still compare to the one in the original paper. We then introduce abundances for each EM, accross the HI.

We consider a HI $B \in \mathbb{R}^{N \times M \times S}$ consisting of $N \times M = \Pi$ pixels, each pixel giving a spectrum evalutated at wavelenghts $w_1,\dots, w_S$. We assume that there are $P$ EMs in the HI, characterized by sets of parameters $\theta_1,\dots, \theta_P$. As we put ourselves in the dispersion model, each $\theta_p \in \mathbb{R}^{N \times M \times K \times 4}$ actually corresponds to the parameters $\rho, \omega_0, \gamma, \epsilon_r$ introduced in the previous section, with different values for different pixels (for skae of lisibility, we define $T = K \times 4$ the number of parameters per EM). Then, we define the matrix of abundances $A \in [0, 1]^{N \times M\times P}$ such that $a_{i, j, p}$ corresponds to the abundance of EM $p$ in pixel $(i, j)$, and $\|A_{i, j, 1:P}\|_1 = 1$. Finally, $E(\theta_1,\dots, \theta_P) = E(\Theta) \in \mathbb{R}^{N \times M \times P \times S}$ gives the reference spectrum of each EM at each pixel, derived from the dispersion model: $E(\Theta)_{i, j, p, s} = \epsilon(\theta_{i, j, p}; w_s)$.

The reconstruction problem we then want to solve states as follows:
\begin{equation}
  \label{eq:reconstruction}
  \argmin_{\Theta \in \mathbb{R}^{\Pi \times P \times K \times 4}, A \in \mathbb{R}^{\Pi \times P}} H(\Theta, A)
\end{equation}
such that $A \geq 0, \|A_{i, j, 1:P}\|_1 = 1$ for all $1\leq i \leq N$, $1\leq j \leq M$, 

with $H(\Theta, A) = \frac{1}{2} \|B - E(\Theta) \cdot A\|_{F, 2}^2$. The norm defined by $\|Q\|_{F, 2} = \sqrt{\sum_{1 \leq i \leq N,\\1 \leq j \leq M} \|Q_{i, j, 1:S}\|_2^2}$ is the Froebenius norm applied on the square norm over spectrum for every pixel. The motivation is to promote small differences between the reconstruction spectra and the target HI, all over the pixels.

\subsection{Constraints}
The elements introduced in the previous section must respect some constraints: the abundances must obviously be positive and sum to 1 in each pixel, while the $\theta$ parameters should stay in a range of values, given their physics meaning.

Regarding the latter, we will accuretely follow what has been done in Janiczek. In particular, we hope that the initialisation and the model will be good enough so that the parameters won't diverge away from reasonable values. Otherwise, we shall project the parameters on the accepted space.

The abundances constraints can be formulated by an indicator operator:
$$\iota_\Upsilon (A) = \begin{cases} 0 \text{ if A}\in \Upsilon\\ +\infty \text{ otherwise} \end{cases}$$
where
$$\Upsilon = \{Q \in \mathbb{R}^{\Pi \times P}_+\, |\,\| Q_{i, j, 1:P} \|_1 = 1 \ \forall 1 \leq i \leq N, 1 \leq j \leq M\}$$
is the simplex where abundances live.

However, this constraint introduces a significant discontinuity in the problem, then losing differentiability. For optimization reasons that are detailed later on, we chose to relax the constraints into penalty terms.

The idea is to strongly promote values close to $\Upsilon$ for $A$, thanks to an approximation of the indicator. Such approach has already been presented and used in the literature, in the form of log-barrier method. In particular, we chose the \emph{log-barrier extension} introduced in Kervadec2020. This operator is convex, continuous and twice-differentiable, and is defined as follows, for some hyper-parameter $t > 0$:
$$\forall z\in \mathbb{R},\ \tilde\Psi_t (z) = \begin{cases} - \frac{1}{t} \log (-z) & \text{if } z \leq -\frac{1}{t^2}\\ tz - \frac{1}{t} \log(\frac{1}{t^2}) + \frac{1}{t^2} & \text{otherwise} \end{cases}$$
The higher $t$, the better the quality of the approximation; however one should be cautious not to overwhelm the other terms in the problems - this is discussed the next section.

To translate the sum-to-one constraint on the abundances, we rewrite it as a sum-greater-than-one and sum-lesser-than-one constraint, in addition to the positivity one. Finally, we come up with the following penalty terms:
\begin{equation}\label{eq:g_c}
\begin{split}
  g_c (A) = & \sum_{i = 1}^N \sum_{j = 1}^M \sum_{p=1}^P \tilde\Psi_\nu (-a_{i,j,p})\\
  &+ \sum_{i = 1}^N \sum_{j = 1}^M \left[ \tilde\Psi_\nu(\sum_{p=1}^P a_{i,j,p} - 1) + \tilde\Psi_\nu(1 - \sum_{p=1}^Pa_{i,j,p}) \right]
\end{split}
\end{equation}

\subsection{Smoothness priors}
In this section are explained the priors introduced in the model.

We would like to enforce spatial regularization on $\Theta$, to represent the idea that, as the environmental conditions are roughly continuous with respect to space dimensions, so are material states. For instance, the quantity of water received by a lawn is close to the one received by the adjacent grass. Obviously, this hypothesis is not always appropriate, as your neighbor might water his lawn way much than you; but it should still be a relevant hypothesis in many applications.

\begin{equation}\label{eq:f}
  f(\Theta) = \gamma \mathcal{L}(E(\Theta))
\end{equation}

where, for every $Y \in \mathbb{R}^{N \times M \times \mathcal{X}}$ which can be seen as a ``collection'' of $N \times M$ images over some domain indexed by $\mathcal{X}$:

\begin{align}\label{eq:smoothing}
  \mathcal{L}(Y) &= \| D(Y) \|_{1,2,\mathcal{F}}^2\\
  &= \| \left( \| (u_h (Y_x), u_v (Y_x)) \|_{1,2} \right)_{x \in \mathcal{X}} \|_{\mathcal{F}}^2\\
  &= \left\| \left(\sum_{i=1}^N \sum_{j=1}^M \sqrt{ u_h (Y_{x})[i, j]^2 + u_v (Y_x)[i, j]^2 } \right)_{x \in \mathcal{X}} \right\|_{\mathcal{F}}^2\\
  &= \sum_{x \in \mathcal{X}} \left( \sum_{i=1}^N \sum_{j=1}^M \sqrt{ u_h (Y_{x})[i, j]^2 + u_v (Y_x)[i, j]^2 } \right)_{x \in \mathcal{X}}^2
\end{align}

and, for an $N \times M$ image $I$:

\begin{align}\label{eq:discrete-grad}
  u_h (I) [i, j] &= I_{i+1, j} - I_{i, j} \text{ if }i + 1 \leq N,\ 0 \text{ otherwise}\\
  u_v (I) [i, j] &= I_{i, j+1} - I_{i, j} \text{ if }j + 1 \leq M,\ 0 \text{ otherwise}
\end{align}

are the discrete \emph{horizontal} and \emph{vertical} gradients.

(\ref{eq:f}) use some regularization term $\gamma \geq 0$, and a smoothing operator $\mathcal{L}$, directly inspired by the \emph{Total Variation} measure (see Condat 2014), applied for each EM.

\subsection{Sparsity priors}

Regarding the abundances, we whish to enforce sparsity of abundances in every pixel; we would rather have a few EMs in significant abundances than a lot of EMs in small quantities. Common choices to do so are the $\ell_0$ count measure, the $\ell_1$ norm, or the $\ell_p \ (p<1)$ quasi-norm. However, these terms are hard to handle during optimisation, as they introduce non-convexity and discontinuity. Hence, to keep once again our problem differentiable, we ressort to the SPOQ regularization, introduced by Cherni in 2020. It gives a smooth approximation of $\ell_0$.

For some parameters $p\in \, ]0, 2[$, $q\in \, [2, +\infty[$, and $(\alpha, \eta) \in \, ]0, +\infty [^2$, this term is defined as follow, for $x \in \mathbb{R}^N$:
$$SPOQ(x) = \log \left( \frac{(\ell_{p,\alpha}^p (x) + \beta^p)^{1/p}}{\ell_{q,\eta} (x)} \right)$$
        where
        $$\ell_{p,\alpha} (x) = \left( \sum_{n=1}^N (x_n^2 + \alpha^2)^{p/2} - \alpha^p \right)^p$$
        $$\ell_{q,\eta} (x) = \left( \eta^q + \sum_{n=1}^N |x_n|^q \right)^{1/q}$$

The choice of the parameters is discussed in the next section.

We then apply the SPOQ pseudo-norm on the vector of abundances in every pixel of the HI:
$$SPOQ(A) = (SPOQ(a_{i,j,1:P}))_{i, j} \in \mathbb{R}_+^{N \times M}$$

In the end, we define the sparsity term on abundances as:
\begin{equation}\label{eq:g_s}
  g_s(A) = \frac{\delta}{N \times M} \| SPOQ(A) \|_{1}
\end{equation}

We simply sum the SPOQ norms over the pixels.

\subsection{Optimization problem}
In this section we wrap-up what have been said so far, to first come up with what would be our ideal model, and secondly we relax it into a differentiable one, as an approximation.
\begin{equation}
  \label{eq:ideal}
  \argmin_{\Theta \in \mathbb{R}^{\Pi \times P \times K \times 4}, A \in \mathbb{R}^{\Pi \times P}} H(\Theta, A) + f(\Theta) + \delta \|A\|_{1,0} + \iota_\Upsilon (A)
\end{equation}
where $\|A\|_{1,0} = \sum_{i,j,p} \mathbf{1}_{(a_{i,j,p} > 0)}$. As disclaimed above, this objective function is highly non-continuous, and while there are some methods to handle such difficulties we didn't find a fitting solution in our case. This is further discussed in the following section.

Hence, we use the approximation terms described earlier to relax the objective function into this differentiable one:

\begin{equation}
  \label{eq:general}
  \argmin_{\Theta \in \mathbb{R}^{\Pi \times P \times K \times 4}, A \in \mathbb{R}^{\Pi \times P}} H(\Theta, A) + f(\Theta) + g(A)
\end{equation}
where $g(A) = g_c(A) + g_s(A)$ (see (\ref{eq:g_c}) and (\ref{eq:g_s})).


\section{Inference algorithm}

The optimisation problem presented in (\ref{eq:ideal}) is composed of one differentiable, non-convex block $H(\Theta, A) + f(\Theta)$ and one non-convex, non-continuous block $\delta\|A\|_{1,0} + \iota_\Upsilon (A)$, and deal with two vectors of parameters $\Theta$ and $A$.

The first alternative to non-differentiability is to use Moreau's proximal operator: a first-order approximation of a gradient descent step. We would then operate an alternating minimization on $\Theta$ (differentiable) and $A$ (non-differentiable). However, algorithms involving this operator typically rely on the knowledge of a (tight) Lipschitz constant of the differentiable part to derive a fitting learning rate, and due to the complex definition of the dispersion model we do not have access to such a constant. Moreover, one can't easily derive the close form of the proximal operator of an addition of two non-differentiable, non-convex terms.

Related families of optimization algorithms would be subgradient or bundle algorithms, unfortunately they do not seem to apply to our problem, mainly because of the non-convexity and the fact that in the constraint $\iota_\Upsilon$ on $A$, $\Upsilon$ is a simplex rather than a convex space.

Hence, instead of testing such approaches without guarantees, we chose a Lagrangian-based method from Kervadec2020, using a \emph{log-barrier extension} to replace the constraint term for $A$. Then, thanks to Cherni2020 we approximate the sparsity term on $A$ with a smooth approximation, so that as written in (\ref{eq:general}) we put ourselves in a smooth context.

Finally, thanks to these rewritings we are able to do a Stochastic Gradient Descent-like optimisation with Adam-which has already shown to be an efficient approach for such problems. A discussion can be made on other Adam-like algorithms in a non-convex problem (cf Chen2019), yet we will stick to the classical Adam for simplicity reasons. In particular, we shall use the Autodiff feature from PyTorch rather than deriving by hand the awfully complex gradient of (\ref{eq:general}).

To wrap-up, the inclusion of our priors raised some major optimization issues, due to which we have been forced to resort to a sub-optimal approach. Hopefully, the behavior will still be good, thanks to the senseful priors and a good initialization provided by the dispersion model. On the other hand, Janiczek \emph{analysis-by-synthesis} consists in an alternating minimization of the reconstruction error, on the abundances and the physics parameters. The main difference with our approach is that the parameters (and then the signature spectrum) of EMs are assumed to be the same all accross the HI.

\section{Experiments}
\subsection{Experimental settings}
\textbf{Datasets:} Describe datasets used for evaluation of the algorithm. Compare with the ones from Janiczek paper.

\textbf{Evaluation criteria:} RMSE, ASAM, MSE for abundances (per EM and for all), reconstruction error $H(\Theta, A)$. Compare with Janiczek.

\textbf{Implementation and settings:} how the code has been implemented, github link, machine on which it ran. Hyperparameters: learning rate, constraints bounds...

\subsection{Results and Discussion}
Present result tables, images (ground truth/Janiczek/us), signatures (mean, max and min against ground truth and Janiczek).


\section{Conclusion}
\textbf{Future work:} use something better than Linear Unmixing.

%\section*{Acknowledgment}

\acks{Put acknowledgements here}

%\section*{References}

% \bibliographystyle{IEEEtran}
\bibliography{unmixing}


\section{Appendix}

\end{document}
