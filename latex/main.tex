\documentclass[conference]{IEEEtran}
\IEEEoverridecommandlockouts
% The preceding line is only needed to identify funding in the first footnote. If that is unneeded, please comment it out.
\usepackage{cite}
\usepackage{amsmath,amssymb,amsfonts}
\usepackage{algorithmic}
\usepackage{graphicx}
\usepackage{textcomp}
\usepackage{xcolor}
\def\BibTeX{{\rm B\kern-.05em{\sc i\kern-.025em b}\kern-.08em
    T\kern-.1667em\lower.7ex\hbox{E}\kern-.125emX}}
\begin{document}

\title{Conference Paper Title*\\
{\footnotesize \textsuperscript{*}Note: Sub-titles are not captured in Xplore and
should not be used}
}

\author{\IEEEauthorblockN{Aymeric CÔME}
\IEEEauthorblockA{\textit{Master Data Science student} \\
\textit{Université de Lille, Centrale Lille, IMT Lille-Douai}\\
Lille, France \\
aymeric.come.etu@univ-lille.fr}
\and
\IEEEauthorblockN{Pierre-Antoine THOUVENIN}
\IEEEauthorblockA{\textit{Assistant professor and member of SigMA team} \\
\textit{Université de Lille, CNRS, Centrale Lille, CRIStAL}\\
Lille, France \\
pierre-antoine.thouvenin@centralelille.fr}
}

\maketitle

\begin{abstract}
  Hyperspectral unmixing is the task of infering the abundance of pure materials in an image from the observed spectra. The resulting spectrum indeed depends on some characteristical features of the End-Members (EMs), however this is a challenging problem because not only it is nonlinear, but also miscellaneous factors can change the output, which is known as Spectral Variability (SV). To tackle this problem, a paper~\cite{janiczek_differentiable_2020} introduced a physics-based, differentiable model to realistically capture EM HV, and then used it in an optimization framework in order to find promising results for spectral unmixing. However, while the dispersion model seems fruitful, the optimization part raises questions. In the following is presented how we try to improve the unmixing process.
\end{abstract}

\begin{IEEEkeywords}
Hyperspectral Unmixing, Spectral Unmixing, Alternating Minimization, Proximal
\end{IEEEkeywords}

\section{Introduction}
In many real-world applications like space exploration, medicine, environment monitoring, mineral detection, hyperspectral images (HI) (consisting of samples of electromagnetic spectra at various wavelengths) can put into evidence some key characteristics of the location monitored. In particular, it is commonly supposed that each present EM have its own signature spectrum, which contributes to the HI depending on its abundance in each pixel. Hence, being able to unmix these characteristic signatures from HI leads to a precise detection of materials in a scene.

However, despite the high spectral resolution of hyperspectral cameras, there are technical limits that must be taken into account; in particular low spatial resolution is to be expected, which leads to a mixture of EMs in each piel. While many unmixing algorithms already have been proposed, the physics behind is often too complex to be fully modelled.

A first difficulty to handle is the SV: depending on the scene observed, an EM might have a different signature than what we were expecting from previous observations. This variability can even occurs locally, between pixels of a same HI, which actually question the hypothesis of one signature corresponding to one EM, rather than a class of signatures. Multiple factors can contribute to this variability, and we distinguish two kinds: intrisic and external. Intrisic variability is due to the physical differences of the EM samples across the HI (e.g. grass is greener on the other side because the neighbor water it) while external variability refers to the environment influence, like light and shade or the quality of the hyperspectral camera used.

In the end, the unmixing problem is a very complex one, and requires advanced technics to overcome the difficulties. We will present some of them in the following section.
\section{Related work}
\subsection{Linear models}

\subsection{Spectral variability}

\subsection{Unsupervised unmixing}

\subsection{Alternating minimization, proximal operators}

\section{Dispersion model}
Present~\cite{janiczek_differentiable_2020} work and model.

\section{Alternating minimization with proximal operators}
Present our unmixing algorithm.

\section{Experimentations}

\section{Conclusion}

\section*{Acknowledgment}

\section*{References}

\bibliographystyle{IEEEtran}
\bibliography{unmixing}

\end{document}
